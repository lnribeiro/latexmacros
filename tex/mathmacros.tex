% Do not forget to use packages amsmath, amssymb, mathtools, bm, gensymb
\newcommand{\mc}[1]{\ensuremath{\mathcal{#1}}}				% calligraphic letters
\newcommand{\bmm}[1]{\ensuremath{\mathbb{#1}}}				% blackboard letters
\newcommand{\tran}{\mathsf{T}}						% transpose operator
\newcommand{\hermit}{\mathsf{H}}					% hermitian operator
\newcommand{\frob}[1]{\ensuremath{\left\|#1\right\|_\textrm{F}}}	% frobenius norm
\newcommand{\neuclid}[1]{\ensuremath{\left\|#1\right\|_\textrm{2}}}	% l2 norm
\newcommand{\esp}[1]{\ensuremath{\mathbb{E}\left[#1\right]}}		% stat expectation
\newcommand{\what}[1]{\ensuremath{\widehat{#1}}}
\newcommand{\wtil}[1]{\ensuremath{\widetilde{#1}}}
\newcommand{\wbar}[1]{\ensuremath{\overline{#1}}}

\DeclareMathOperator{\Blkdiag}{Blkdiag}
\DeclareMathOperator{\Diag}{Diag}
\DeclareMathOperator{\diag}{diag}
\DeclareMathOperator{\trace}{Tr}
\DeclareMathOperator{\rank}{rank}
\DeclareMathOperator{\vecspan}{span}
\DeclareMathOperator{\matnull}{null}

\let\vec\relax
\DeclareMathOperator{\vec}{vec}
\DeclareMathOperator{\unvec}{unvec}
\DeclarePairedDelimiterX{\openinterval}[2]{]}{]}{#1,#2}
\DeclarePairedDelimiterX{\openintervalboth}[2]{]}{[}{#1,#2}
\DeclarePairedDelimiterX{\closedinterval}[2]{[}{]}{#1,#2}